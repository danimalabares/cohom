\documentclass[spanish]{book}
\usepackage{titlesec}

%Quitar páginas en blanco
\let\cleardoublepage\clearpage
\usepackage{etoolbox}
\makeatletter
\patchcmd{\@endpart}{\vfil\newpage}{\par}{}{}
\makeatother

%\usepackage[spanish]{babel} ¡Esto estaba interfiriendo con las flechitas de los \tikspicture

\renewcommand{\contentsname}{Índice}
\renewcommand{\partname}{Parte}

\titleformat{\chapter}[display]
{\normalfont\huge\bfseries}{}{0pt}{\Huge\thechapter.~}

\titleformat{name=\chapter,numberless}[display]
{\normalfont\huge\bfseries}{}{0pt}{\Huge}
\renewcommand{\chaptermark}[1]{\markboth{{} \thechapter: #1}{}}

\usepackage[left=4cm, right=4cm]{geometry}
\usepackage{palatino}%Font
\usepackage{graphicx}
\usepackage{marvosym}%Smileys
\usepackage{float}
\usepackage{subcaption}
\usepackage{enumitem}
\usepackage{parskip}
\usepackage{amsthm}
\usepackage{amssymb}
\usepackage{amsmath}
\usepackage{stmaryrd}
\usepackage{tikz}
\usepackage{tikz-cd}
\usepackage[bookmarks,bookmarksopen,bookmarksdepth=3]{hyperref}
\hypersetup{
	colorlinks=true,
	urlcolor=blue,
	linkcolor=magenta,
	citecolor=blue,
	filecolor=blue,
	urlbordercolor=white,
	linkbordercolor=white,
	citebordercolor=white,
	filebordercolor=white
}

\theoremstyle{definition}
\renewcommand{\proofname}{Demostración}

\newtheorem*{defn}{Definición}
\newtheorem*{lema}{Lema}
\newtheorem*{obs}{Observación}
\newtheorem*{teo}{Teorema}
\newtheorem*{prop}{Proposición}
\newtheorem*{coro}{Corolario}
\newtheorem*{ejer}{Ejercicio}
\newtheorem*{ejem}{Ejemplo}
\newtheorem*{af}{Afirmación}
\newtheorem*{pregunta}{Pregunta}

\newcommand{\R}{\mathbb{R}}
\newcommand{\Z}{\mathbb{Z}}
\newcommand{\N}{\mathbb{N}}
\newcommand{\C}{\mathbb{C}}
\newcommand{\Q}{\mathbb{Q}}
\DeclareMathOperator{\img}{img}
\DeclareMathOperator{\coker}{coker}
\DeclareMathOperator{\ch}{ch}
\DeclareMathOperator{\comp}{comp}
\DeclareMathOperator{\RCH}{R\text{-}ch\text{-}comp}
\DeclareMathOperator{\RMod}{R\text{-}mod}
\DeclareMathOperator{\ZMod}{\mathbb{Z}\text{-}mod}
\DeclareMathOperator{\Ab}{Ab}
\DeclareMathOperator{\Hom}{Hom}
\DeclareMathOperator{\Cohom}{Cohom}
\DeclareMathOperator{\Ext}{Ext}



\title{La conjetura de Poincaré en dimensiones altas}
\author{Notas\\ \\ \href{https://github.com/danimalabares/cohom}{github.com/danimalabares/cohom}}

\begin{document}
	\maketitle
	\phantomsection
	\addcontentsline{toc}{part}{\contentsname}
	\tableofcontents
	
\chapter{Ágebra homológica}
\section{Repaso}
Sea $R$ un anillo asociativo con 1. Podemos ahora tomar la categoría de $R$-módulos, $R$-$\mod$, cuyos objetos son $R$-módulos y los morfismos son homomorfismos $R$-lineales. También podemos construir $\RCH$, cuyos objetos son complejos de cadenas,
\[\begin{tikzcd}
	\cdots\arrow[r,"\partial_{n+1}"]&C_n\arrow[r,"\partial_{n}"]&C_{n-1}\arrow[r,"\partial_{n-1}"]&\cdots
\end{tikzcd}\]
tales que $\partial_{n-1}\circ\partial_n=0$, es decir, $\img\partial_n\subseteq\ker\partial_{n-1}$.
y sus morfismos son morfismos complejos de cadenas, \begin{tikzcd}
	C_\bullet\arrow[r,"f"]&D_\bullet\end{tikzcd},
que son muchos morfismos tales que el siguiente diagrama conmuta en todos los cuadraditos:
\[\begin{tikzcd}
	\cdots \arrow{r} & C_{n+1} \arrow{r}{\partial_{n+1}} \arrow{d}{f_{n+1}} & C_n \arrow{r}{\partial_n} \arrow{d}{f_n} & C_{n-1} \arrow{r} \arrow{d}{f_{n-1}} & \cdots \\
	\cdots \arrow{r} & D_{n+1} \arrow{r}{\delta_n} & D_n \arrow{r}{\delta_n} & D_{n-1} \arrow{r} & \cdots
\end{tikzcd}\]
Y definimos
\[H_n(C_\bullet)=\frac{\ker\partial_n}{\img\partial_{n+1}}\]
\begin{defn}\leavevmode
	\begin{itemize}
		\item Decimos que $C_\bullet$ es \textbf{acíclico} si $H_n(C_\bullet)=0$ para toda $n$.
		\item La sucesión
		\begin{tikzcd}[column sep=small]
			C_1\arrow[r,"\varphi"]&C_2\arrow[r,"\psi"]&C_3
		\end{tikzcd}
		es \textbf{exacta} si $\img\varphi=\ker\psi$.
		\item La sucesión 
		\begin{tikzcd}[column sep=small]
			0\arrow[r]&C_1\arrow[r]&C_2\arrow[r]&C_3\arrow[r]&0
		\end{tikzcd}
		es una \textbf{sucesión exacta corta}.
		\item Y si se extiende infinitamente, es una \textbf{sucesión exacta larga}.
	\end{itemize}
\end{defn}
\begin{prop} En una sucesión exacta corta, $\varphi$ es inyectiva, $\psi$ es suprayectiva y $C_3\approx C_2/\ker\psi$. Abusando de notación, podemos pensar que $C_3\approx C_2/C_1$, pero hay que tener cuidado aquí porque el encaje de $C_1$ en $C_2$ puede no ser único.
\end{prop}
Tomemos $n$ fijo. Entonces
\[\begin{tikzcd}
	&\Ab\\
	\RCH \arrow[r,"H_n"]\arrow[ru]&\RMod \arrow[u]\\[-0.6cm]
	C_\bullet\arrow[r,maps to]&H_n(C_\bullet)
\end{tikzcd}\]
Y como los morfismos de cadenas mandan ciclos en ciclos y fronteras en fronteras, podemos definir los morfismos inducidos, que satisfacen que $(fg)_*=f_*g_*$ y $id_{C_{\bullet*}}=id_{H_n(C_\bullet)}$. Como la composición de morfismos se abre en el mismo orden en el que estaba, se llama \textbf{funtor covariante}.
\begin{defn}
	Dos homomorfismos 
	\begin{align*}
		f,g:(C_\bullet,\partial)\to(C'_\bullet,\partial')
	\end{align*}
	son \textbf{homotópicos} si existen homomorfismos $h_n:C_n\to C'_{n+1}$ para toda $p\in\Z$ tales que \[f_n-g_n=\partial'_{n+1}h_n+h_{n-1}\partial_n\] Estas flechas se pueden visualizar aquí:
	\[\begin{tikzcd}[column sep=large, row sep=large]
		\cdots \arrow{r} & C_{n+1} \arrow{r}{\partial_{n+1}} \arrow{d}[left]{f_{n+1}-g_{n+1}} & C_n \arrow{r}[blue]{\partial_n} \arrow{d}[right,red]{f_n-g_n} \arrow{ld}[left,blue]{h_n} & C_{n-1} \arrow{r} \arrow{d}[right]{f_{n-1}-g_{n-1}} \arrow{ld}[right,blue]{h_{n-1}} & \cdots \\
		\cdots \arrow{r} & C'_{n+1} \arrow{r}[below,blue]{\partial'_{n+1}} & C'_n \arrow{r}[below]{\partial_n'} & C'_{n-1} \arrow{r} & \cdots
	\end{tikzcd}
	\]
	Así que la suma de las flechas azules es igual a la flecha roja. (No estamos diciendo que el diagrama sea conmutativo).
	
	Esto es tanto como decir que $H_n(f)=H_n(g)$ para toda $n$. Es decir, funciones homotópicas inducen los mismos homomorfismos entre complejos de cadenas.
\end{defn}
	\begin{teo}[fundamental del álgebra homológica]
	Si 
	\[\begin{tikzcd}[column sep=small]
		0\arrow{r}&A_\bullet\arrow{r}{\phi}&B_\bullet\arrow{r}{\psi}&C_\bullet\arrow{r}&0
	\end{tikzcd}\]
	es una sucesión exacta corta de complejos de cadena, entonces existen homomorfismos \[\delta_{*p}:H_p(C_\bullet)\to H_{p-1}(A_\bullet)\]
	tales que la sucesión
	\[\begin{tikzcd}[column sep=small]
		&\cdots\arrow{r}&H_p(A_\bullet)\arrow{r}{\bar\phi_p}&H_p(B_\bullet)\arrow{r}{\bar\psi_p}&H_p(C_\bullet)\arrow{r}{\delta_{*p}}&H_{p-1}(A_\bullet)\arrow{r}{\bar\phi_{p-1}}&H_{p-1}(B_\bullet)\arrow{r}&\cdots
	\end{tikzcd}\]
	es exacta.
\end{teo}
En el siguiente diagrama conmutativo se ve claramente qué está pasando:
\[
\begin{tikzcd}
	& & 0 \arrow{d} & 0 \arrow{d} & 0 \arrow{d} & \\
	& \cdots \arrow{r} & A_{p+1} \arrow{r}{\partial_{p+1}} \arrow{d}{i_{p+1}} & A_p \arrow{r}{\partial_p} \arrow{d}{i_p} & A_{p-1} \arrow{r} \arrow[d,magenta,"i_{p-1}"] & \cdots \\
	& \cdots \arrow{r} & B_{p+1} \arrow{r}{\partial_{p+1}} \arrow{d}{j_{p+1}} & B_p \arrow[r,magenta,"\partial_p"] \arrow[d,magenta,"j_p"] & B_{p-1} \arrow{r} \arrow{d}{j_{p-1}} & \cdots \\
	& \cdots \arrow{r} & C_{p+1} \arrow{r}{\partial_{p+1}} \arrow{d} & C_p \arrow{r}{\partial_p} \arrow{d} & C_{p-1} \arrow{r} \arrow{d} & \cdots \\
	& & 0 & 0 & 0 & \\
\end{tikzcd}
\]
\begin{proof}
	Explicamos un poco cómo definir el homomorfismo de conexión haciendo cacería de diagrama. Comenzamos con un ciclo $c\in C_p(A)$. Como $j_p$ es suprayectiva, existe un $a\in B_p$ tal que $j_p(a)=c$. Luego, $\partial_p(a)\in\ker j_{p-1}$, ya que, como el diagrama conmuta, $\partial_pj_p=j_{p-1}\partial_p$ y $c$ es un ciclo. Como la sucesión es exacta, $\ker j_{p-1}=\img i_{p-1}$, así que existe $a\in A_{p-1}$ tal que $i_{p-1}(a)=\partial_p(b)$. Este $a$ es un ciclo, ya que el diagrama conmuta, $i_{p-2}(a)=\partial(\partial(b))=0$, y la $i_{p-2}$ es inyectiva por exactitud, es decir, el único elemento al que va a dar el cero es el cero. Así que definimos $\delta_{*p}[c]=[a]$.
	
	Y una vez definido este homomorfismo, el resto de la prueba sale sin trucos.
\end{proof}
\begin{teo}[Naturalidad del homomorfismo de conexión]
	Para dos sucesiones exactas cortas y morfismos $f$, $g$ y $h$,
	\[\begin{tikzcd}[column sep=small]
		&0\arrow{r}&A_\bullet\arrow{r}{i}\arrow{d}{f}&B_\bullet\arrow{r}{j}\arrow{d}{g}&C_\bullet\arrow{r}\arrow{d}{h}&0\\
		&0\arrow{r}&A'_\bullet\arrow{r}&B'_\bullet\arrow{r}&C'_\bullet\arrow{r}&0
	\end{tikzcd}\]
	donde las filas son exactas.\par
	Entonces, el siguiente diagrama conmuta
	\[\begin{tikzcd}[column sep=small, row sep=large]
		& \cdots \arrow{r} & H_p(A) \arrow{r} \arrow{d}{\bar{f}} & H_p(B) \arrow{r} \arrow{d}{\bar{g}} & H_p(C) \arrow{r}{\delta_*} \arrow{d}{\bar{h}} & H_{p-1}(A) \arrow{r} \arrow{d}{\bar{f}} & H_{p-1}(B) \arrow{r} \arrow{d}{\bar{g}} & H_{p-1}(C) \arrow{r} \arrow{d}{\bar{h}} & \cdots \\
		& \cdots \arrow{r} & H_p(A') \arrow{r} & H_p(B') \arrow{r} & H_p(C') \arrow{r} & H_{p-1}(A') \arrow{r} & H_{p-1}(B') \arrow{r} & H_{p-1}(C') \arrow{r} & \cdots
	\end{tikzcd}\]
\end{teo}
\begin{proof}
	Salvo en los cuadrados donde está $\bar{h}$ a la izquierda y $\bar{f}$ a la derecha, la conmutatividad se sigue por funtorialidad.
\end{proof}
\begin{lema}[de los cinco]
	Consideremos el diagrama conmutativo con filas exactas
	\[\begin{tikzcd}
		M_5\arrow{r}{f_5}\arrow{d}{h_5}&M_4\arrow{r}{f_4}\arrow{d}{h_4}&M_3\arrow{r}{f_3}\arrow{d}{h_3}&M_2\arrow{r}{f_2}\arrow{d}{h_2}&M_1\arrow{d}{h_1}\\
		N_5\arrow[r,swap,"g_5"]&N_4\arrow[r,swap,"g_4"]&N_3\arrow[r,swap,"g_3"]&N_2\arrow[r,swap,"g_2"]&N_1
	\end{tikzcd}\]
	Si $h_5,h_4,h_2$ y $h_1$ son isomorfismos, entonces $h_3$ también.
\end{lema}

\section{Más álgebra homológica}
Tomemos $N,M$ $R$-módulos, y el conjunto de homomorfismos $R$-lineales de $M$ en $N$, que es un grupo abeliano (cuya identidad es el morifsmo que manda todo a $0$, y $f+g(m)=f(m)+g(m)$ que también es un morfismo, $(-f)(m)=-f(m)$). También tiene estructura de $R$-módulo con la operación $(rf)(m)=rf(m)=f(rm)$.

Ahora construyamos un funtor:
\begin{align*}
	\Hom(-,N):\RMod&\to\Ab\\
	M&\mapsto\Hom(M,N)\\
	M\xrightarrow{\varphi} M'&\mapsto \qquad?
\end{align*}
La flecha inducida será
\begin{align*}
	\Hom(M,N)&\xleftarrow{\varphi^*}\Hom(M',N)\\
	\varphi^*(f)=f\varphi&\mapsfrom f
\end{align*}
De acuerdo a
\[\begin{tikzcd}
	M\arrow[d,"\varphi"]\arrow[rd,"f\varphi"]\\
	M'\arrow[r,swap,"f"]&N
\end{tikzcd}\]
Así que $\Hom(-,N)$ es un funtor \textbf{contravariante}. De hecho, es un \textbf{funtor aditivo exacto izquierdo}:
\begin{itemize}
	\item \textbf{Aditivo}. Manda sumas directas en sumas directas, es decir,
	\[\Hom(M_1\oplus M_2,N)\approx \Hom(M_1,N)\oplus\Hom(M_2,N)\]
	Que tiene que ver con la propiedad universal de la suma directa:
	\[\begin{tikzcd}
		M_1\arrow[r,hook]\arrow[rd,"f"]&M_1\oplus M_2\arrow[d,dashed,"\exists! f\oplus g"]&M_2\arrow[l,hook']\arrow[dl,"g"]\\
		&N
	\end{tikzcd}\]
	Donde $(f\oplus g)(m_1,m_2)=f(m_1)+g(m_2)$. Así que si tenemos $(f,g)$ en el módulo de la derecha, lo mandamos a $f\oplus g$.
	\item \textbf{Exacto} Supongamos que tenemos la sucesión exacta corta
	\[\begin{tikzcd}
		0\arrow[r]&A\arrow[r,"\varphi"]&B\arrow[r,"\psi"]&C\arrow[r]&0
	\end{tikzcd}\]
	a la que le aplicamos el funtor para obtener la sucesión exacta
	\[\begin{tikzcd}
		0\arrow[r]&\Hom(C,N)\arrow[r,"\psi^*"]&\Hom(B,N)\arrow[r,"\varphi^*"]&\Hom(A,N)
	\end{tikzcd}\]
	En general, $\varphi^*$ no es suprayectiva.
\end{itemize}

\begin{ejer}\leavevmode
	\begin{itemize}
		\item Checar lo anterior.
		\begin{proof}[Solución]
			Basta ver que la flecha $0\to A$ no necesiamente va a dar a una flecha de la forma $\Hom(A,N)\to0$.
		\end{proof}
		\item ¿Qué pasa con el cokernel?
	\end{itemize}
\end{ejer}
\begin{obs}
	También podemos definir el funtor análogo dejando libre la entrada de la derecha, y obtenemos un funtor covariante (que no usaremos tanto y también es aditivo exacto \textit{izquierdo}).
\end{obs}
\begin{obs}
	Denotaremos $\Hom_R(M,N):=M^*$, y, por si acaso $\Hom(N,M):=M_*$.
\end{obs}

\section{Funtores derivados}
Es un juego, y usaremos $R$-módulos libres, que tienen la ventaja de tener una base. Un \textbf{$R$-módulo libre} es uno de la forma $\bigoplus_{i\in I}R_i$ donde $R_i=R$. Los elementos canónicos son $e_j:=(\delta_{ij})_{i\in I}$, y $\beta:=\{ej\}_{j\in J}$ es una \textbf{base} en cuanto a que cumple la siguiente propiedad universal: para cualquier $R$-módulo $M$ y para toda función $f:\beta\to M$ existe un único $\bar{f}:L=\bigoplus_{i\in I}R_i\to M$ tal que el siguiente diagrama conmuta:
\[\begin{tikzcd}
	\beta\arrow[r,"f"]\arrow[d,hook]&M\\
	L\arrow[ur,swap,dashed,"\bar{f}"]
\end{tikzcd}\]
Luego, diremos que $P$ es \textbf{proyectivo} si existe $Q$ tal que $P\oplus Q$ es libre.
\begin{ejem}
	$\Z/6\approx \Z/2\oplus\Z/3$
\end{ejem}
\begin{prop}
	Todo $R$-módulo es cociente de un $R$-módulo libre. 
\end{prop}
\begin{proof}
	\[\begin{tikzcd}
		L=\bigoplus_{i\in M}R_i\arrow[r,dashed,"\bar{f}"]&M\\
		M\arrow[u,hook]\arrow[ur]
	\end{tikzcd}\]
	Como $\bar{f}$ es suprayectiva, por primer teorema de isomorfismo, terminamos.
\end{proof}
\begin{defn}
	Sea $M$ un $R$-módulo. Una \textbf{resolución libre (proyectiva)} de $M$ es una sucesión exacta de la forma
	\[\begin{tikzcd}
		\cdots\arrow[r]&F_1\arrow[r,"f_1"]&F_0\arrow[r,"f_0"]&M\arrow[r]&0
	\end{tikzcd}\]
	tal que $F_j$ es libre para toda $j$.
\end{defn}
\begin{teo}
	Todo $R$-módulo tiene una resolución libre (proyectiva).
\end{teo}
\begin{proof}
	$f_0$ sale por la proposición anterior. Tomamos el módulo $\ker f_0$, lo incluimos en $F_0$ escogemos $F_1$ que cubre $\ker f_0$ por la proposición anterior.
	\[\begin{tikzcd}
		\cdots\arrow[r]&F_1\arrow[rd,"f_1"]\arrow[r,"f_1"]&F_0\arrow[r,"f_0"]&M\arrow[r]&0\\
		&&\ker f_0\arrow[u,hook]&&
	\end{tikzcd}\]
\end{proof}
\begin{teo}
	Sea $\alpha:M\to M'$ un homomorfismo.
	\[\begin{tikzcd}
		\cdots\arrow[r]&F_2\arrow[r,"f_2"]\arrow[d,dashed,"\alpha_2"]&F_1\arrow[r,"f_1"]\arrow[d,dashed,"\alpha_1"]&F_0\arrow[d,dashed,"\alpha_0"]\arrow[r,"f_0"]&M\arrow[d,"\alpha"]\arrow[r]&0\\
		\cdots\arrow[r]&F'_2\arrow[r,"f_2"]&F'_1\arrow[r,"f_1"]&F'_0\arrow[r,"f_0"]&M'\arrow[r]&0
	\end{tikzcd}\]
	entonces existen los $\alpha_i$ que hacen conmutar el diagrama.
	
	Más aún, si existen $\beta_i:F_i\to F'_i$ que cumplen lo mismo entonces los homomorfismos determinados por los $\alpha_i$ y $\beta_i$ son homotópicos.
\end{teo}
\begin{proof}
	\[\begin{tikzcd}
		&\beta_0\arrow[d,hook,]&&\\	
		\cdots\arrow[r]&F_0\arrow[d,dashed,"\alpha_0"]\arrow[r,"f_0"]&M\arrow[d,"\alpha"]\arrow[r]&0\\
		\cdots\arrow[r]&F'_0\arrow[r,"f'_0"]&M\arrow[r]&0
	\end{tikzcd}\]
	Tomamos un elemento $e\in \beta_0$ en la base de $F_0$. Lo mandamos mediante $f_0$ a $M$, luego con $\alpha$. Pero como $f'_0$ es supra, podemos escoger un elemento $e'\in F'_0$ que le pega.
	
	Ahora
	\[\begin{tikzcd}
		&&&\img f_1\arrow[d,hook,]&&\\	
		\cdots\arrow[r]&F_2\arrow[r,"f_2"]\arrow[d,dashed,"\alpha_2"]&F_1\arrow[ur]\arrow[r,"f_1"]\arrow[d,dashed,"\alpha_1"]&F_0\arrow[d,dashed,"\alpha_0"]\arrow[r,"f_0"]&M\arrow[d,"\alpha"]\arrow[r]&0\\
		\cdots\arrow[r]&F'_2\arrow[r,"f'_2"]&F'_1\arrow[dr]\arrow[r,"f'_1"]&F'_0\arrow[r,"f'_0"]&M\arrow[r]&0\\
		&&&\img f'_1\arrow[u,hook,]&&
	\end{tikzcd}\]
	Hay una flecha desde $\img f_1$ hasta $\img f'_1$ que cierra el diagrama.
	
	Faltó lo de homotopía (usando las diagonales como el diagrama coloreado).
\end{proof}
\begin{defn}
	Sea $M$ un $R$-módulo. Tomamos una resolución libre
	\[\begin{tikzcd}
		\cdots\arrow[r]&F_1\arrow[r,"f_1"]&F_0\arrow[r,"f_0"]&M\arrow[r]&0
	\end{tikzcd}\]
	Quitamos $M$:
	\[\begin{tikzcd}
		\cdots\arrow[r]&F_1\arrow[r,"f_1"]&F_0\arrow[r]&0
	\end{tikzcd}\]
	Aplicamos $\Hom_R(-,N)$:
	\[\begin{tikzcd}
		0\arrow[r]&F_0^*\arrow[r,"f_1"]&F_1^*\arrow[r]&F_2^*\arrow[r,"f_3^*"]&\cdots
	\end{tikzcd}\]
	Definimos
	\[\Ext^n_R(M,N):=H_n(0\to F^*)\]
\end{defn}
\begin{teo}
	$\Ext^n_R(M,N)$ no depende de la resolución.
\end{teo}
\begin{proof}
	Usamos el teorema anterior (dos veces), tomando dos resoluciones de $M$ usando la identidad como $\alpha$:
	\[\begin{tikzcd}
		\cdots\arrow[r]&F_2\arrow[r,"f_2"]\arrow[d,"\alpha_2"]&F_1\arrow[r,"f_1"]\arrow[d,"\alpha_1"]&F_0\arrow[d,"\alpha_0"]\arrow[r,"f_0"]&M\arrow[d,"Id"]\arrow[r]&0\\
		\cdots\arrow[r]&F'_2\arrow[r,"f_2"]\arrow[d,"\beta_2"]&F'_1\arrow[r,"f_1"]\arrow[d,"\beta_1"]&F'_0\arrow[d,"\beta_0"]\arrow[r,"f_0"]&M\arrow[d,"Id"]\arrow[r]&0\\
		\cdots\arrow[r]&F_2\arrow[r,"f_2"]&F_1\arrow[r,"f_1"]&F_0\arrow[r,"f_0"]&M\arrow[r]&0
	\end{tikzcd}\]
	Y aquí resulta que $\{\beta_i\alpha_i\}\simeq\{Id\}$. Y dualizamos:
	\[\begin{tikzcd}
		\cdots&F_1^*\arrow[l]&F_0^*\arrow[l]&M^*\arrow[l]&0\arrow[l]\\
		\cdots&F_1^{'*}\arrow[u,"\alpha_1^*"]\arrow[l]&F^{'*}_0\arrow[u,"\alpha_0^*"]\arrow[l]&M^*\arrow[l]&0\arrow[l]\\
		\cdots&F_1^{'*}\arrow[u,"\beta_1^*"]\arrow[l]&F^{'*}_0\arrow[u,"\beta_0^*"]\arrow[l]&M^*\arrow[l]&0\arrow[l]
	\end{tikzcd}\]
	Y como el funtor es aditivo, la homotopía pasa al dual, es decir, $\{\beta_i^*\alpha_i^*\}\simeq\{Id\}$. Luego pasamos a los grupos de homología:
	\[\begin{tikzcd}
		H_1(F^*)\\
		H_1(F'^*)\arrow[u,"\alpha_1^\#"]\\
		H_1(F^*)\arrow[u,"\beta_1^\#"]
	\end{tikzcd}\]
	Cambiando los roles, obtenemos que estas dos funciones $\alpha_1^\#$ y $\beta_1^\#$ son inversas una de la otra.
\end{proof}\vspace{-.3cm}
\begin{prop}
	$\Ext^0_R(M,N)\approx\Hom_R(M,N)$.
\end{prop}
\begin{proof} Tenemos:
	\[\begin{tikzcd}
		0\arrow[r]&\img f_1\arrow[r,"f_1"]&F_0\arrow[r,"f_0"]&M\arrow[r]&0
	\end{tikzcd}\]
	\[\begin{tikzcd}
		0\arrow[r]&M^*\arrow[r,"f_0^*"]&F^*_0\arrow[r,"f_1^*"]&\img f_1^*
	\end{tikzcd}\]
	Luego $\ker f_1^*\approx \img f_0^*\approx M^*$. Luego, por definición $\Ext^0_R(M,N)=\ker f_1^*\approx M^*=\Hom(M,N)$
\end{proof}

\begin{lema}[De la herradura] \textit{Sucesiones exactas cortas de módulos inducen sucesiones exactas cortas de resoluciones.} Tomemos una sucesión exacta corta y dos resulciones libres de los extremos. Entonces existe lo rojo:
	\[\begin{tikzcd}
		&0&0&0\\
		1\arrow[r]&M'\arrow[u]\arrow[r]&M\arrow[red,u]\arrow[r]&M''\arrow[u]\arrow[r]&0\\
		0\arrow[red,r]&F_0'\arrow[r,red]\arrow[u]&\textcolor{red}{F_0}\arrow[r,red]\arrow[u,red]&F_0^{''}\arrow[u]\arrow[r,red]&0\\
		0\arrow[red,r]&F_1'\arrow[r,red]\arrow[u]&\textcolor{red}{F_1}\arrow[r,red]\arrow[u,red]&F_1^{''}\arrow[u]\arrow[r,red]&0\\
		&\vdots\arrow[u]&\vdots\arrow[u,red]&\vdots\arrow[u]&
	\end{tikzcd}\]
\end{lema}
\begin{proof} Pa' pronto, la resolución de en medio es la suma de las resoluciones:
	\[\begin{tikzcd}
		&0&&0\\
		1\arrow[r]&M'\arrow[r]\arrow[u]&M\arrow[r]&M''\arrow[r]\arrow[u]&0\\
		0\arrow[red,r]&F_0'\arrow[r,red]\arrow[u]&F_0'\oplus F_0^{''}\arrow[r,red]\arrow[u,red]&F_0^{''}\arrow[u]\arrow[r,red]&0
	\end{tikzcd}\]
	Y hay que hacer todo lo de rutina.
\end{proof}
Ahora apliquemos $\Hom(-,N)$:
\[\begin{tikzcd}
	&0\arrow[d]&0\arrow[d]&0\arrow[d]&\\
	0&P_0^{'*}\arrow[l]\arrow[d]&P_0^{*}\arrow[l]\arrow[d]&P_0^{''*}\arrow[l]\arrow[d]&0\arrow[l]\\	0&P_1^{'*}\arrow[l]\arrow[d]&P_1^{*}\arrow[l]\arrow[d]&P_1^{''*}\arrow[l]\arrow[d]&0\arrow[l]\\
	&\cdots&\cdots&\cdots&
\end{tikzcd}\]
O sea que tenemos la sucesión exacta corta
\[\begin{tikzcd}
	0&P^{'*}_\bullet\arrow[l]&P^*_\bullet\arrow[l]&P_\bullet^{''*}\arrow[l]&0\arrow[l]
\end{tikzcd}\]
A la que aplicamos el teorema fundamental del álgebra homológica para obtener
\[\begin{tikzcd}
	0\arrow[r]&\Ext_R^0(M'',N)\arrow[r]&\Ext_R^0(M,N)\arrow[r]&\Ext_R^0(M',N)\arrow[r]&\Ext_R^1(M'',N)\arrow[r]&\cdots
\end{tikzcd}\]
Y notemos que los primeros tres módulos son:
\[\begin{tikzcd}
	0\arrow[r]&\Hom_R(M'',N)\arrow[r]&\Hom_R(M,N)\arrow[r]&\Hom_R(M',N)\arrow[r]&\Ext_R^1(M'',N)\arrow[r]&\cdots
\end{tikzcd}\]
Y ahora supongamos que tenemos
\[\begin{tikzcd}
	0\arrow[r]&M'\arrow[r]\arrow[d]&M\arrow[r]\arrow[d]&M''\arrow[r]\arrow[d]&0\\
		0\arrow[r]&A'\arrow[r]&A\arrow[r]&A''\arrow[r]&0
\end{tikzcd}\]
Que inducen
\[\begin{tikzcd}
	0&P^{'*}_\bullet\arrow[l]&P^*_\bullet\arrow[l]&P_\bullet^{''*}\arrow[l]&0\arrow[l]\\
		0&Q^{'*}_\bullet\arrow[l]\arrow[u]&Q^{*}_\bullet\arrow[l]\arrow[u]&Q_\bullet^{''*}\arrow[l]\arrow[u]&0\arrow[l]
\end{tikzcd}\]
Y por fin obtenemos
\[\begin{tikzcd}
	0\arrow[r]&\Ext_R^0(M'',N)\arrow[r]&\Ext_R^0(M,N)\arrow[r]&\Ext_R^0(M',N)\arrow[r]&\Ext_R^1(M'',N)\arrow[r]&\cdots\\
		0\arrow[r]&\Ext_R^0(A'',N)\arrow[r]\arrow[u]&\Ext_R^0(A,N)\arrow[r]\arrow[u]&\Ext_R^0(A',N)\arrow[r]\arrow[r]&\Ext_R^1(M'',N)\arrow[r]\arrow[u]&\cdots
\end{tikzcd}\]
donde todo conmuta. Y ese es más o menos el juego de los funtores derivados.

Es momento de hacer un decreto:
\begin{quotation}
	\textbf{A partir de ahora el anillo será $\Z$.}
\end{quotation}
Tomemos entonces $A,B\in\ZMod$ y el funtor $\Hom_\Z(-,B)$. ¿Cómo será una resolución libre proyectiva para $A$?
\[\begin{tikzcd}
	0\arrow[r]&0\arrow[r]&\ker f_0=P_1\arrow[r]&\bigoplus_{I_0}\Z=P_0\arrow[r,"f_0"]&A\arrow[r]&0
\end{tikzcd}\]
Que inducen
\[\begin{tikzcd}
	0&0\arrow[l]&P_1^*\arrow[l]&P_0^*\arrow[l]&0\arrow[l]
\end{tikzcd}\]
Es decir,
\begin{quotation}
	$\Ext^n_\Z(A,B)=0$ para cualesquiera $\Z$-módulos $A,B$ y $n\geq2$.
\end{quotation}
\begin{prop}\leavevmode
	\begin{itemize}
		\item $\Ext^n_R(-,N)$ es un funtor aditivo para toda $n$, es decir,
		\[\Ext_R^n(M'\oplus M'',N)\approx\Ext_R^n(M',N)\oplus\Ext^n_R(M'',N)\]
		\begin{proof}
			Consideremos
			\[\begin{tikzcd}[row sep=small]
				P'_\bullet\arrow[r]&M'\arrow[r]&0\\
				P''_\bullet\arrow[r]&M''\arrow[r]&0
			\end{tikzcd}\]
			Y no es difícil ver que también tenemos
			\[\begin{tikzcd}
				P'_\bullet\oplus P''_\bullet\arrow[r]&M'\oplus M''\arrow[r]&0
			\end{tikzcd}\]
			Y la homología abre sumas: $(P_\bullet''\oplus P_\bullet'')^*\approx P_\bullet''\oplus P_\bullet'^*$.
		\end{proof}
		\item $\Ext^n_\Z(A,B)=0$ si $A$ es libre.
		\begin{proof}
			Simplemente tomamos $0\to A\to A\to 0$.
		\end{proof}
		\item $\Ext^1_\Z(\Z/n\Z,B)\approx B/nB$.
		\begin{proof}
			\[\begin{tikzcd}[row sep=small]
				0&\Hom(\Z,B)\arrow[l]\arrow[d,"\approx"]&\Hom(\Z,B)\arrow[l,swap,"n^*"]\arrow[d,"\approx"]&0\arrow[l]\\
				0&B\arrow[l]&B\arrow[l,"n"]&0\arrow[l]
			\end{tikzcd}\]
			Así que $B/nB=\Ext_\Z^1(\Z/n,B)$.
		\end{proof}
	\end{itemize}
\end{prop}
¿Qué obtenemos de esta proposición? Si $A$ es finitamente generado, $A\approx\Z^r\oplus\Z/m_1\oplus\ldots\oplus\Z/m_t$, entonces $\Ext^1_\Z(A,B)\approx B/m_1B\oplus\ldots\oplus B/m_tB$.

\section{Grupos de cohomología}
Tomemos un grupo abeliano $G$ y un complejo de cadenas de grupos abelianos libres
\[\begin{tikzcd}
	\cdots\arrow[r]&C_n\arrow[r,"\partial_n"]&C_{n-1}\arrow[r,"\partial_{n-1}"]&C_{n-2}\arrow[r]&\cdots
\end{tikzcd}\]
Y apliquemos el $\Hom_\Z(-,G)$ para obtener
\[\begin{tikzcd}
	\cdots\arrow[r]&C_n^*\arrow[r,"\partial_{n+1}^*"]&C_{n+1}^*\arrow[r,"\partial^*_{n+2}"]&C_{n+2}^*\arrow[r]&\cdots
\end{tikzcd}\]
que tiene su homología,
\[H^n(C_\bullet,G)=\ker\partial_n^*/\img\partial^*_{n-1}\]
que llamaremos el \textbf{$n$-ésimo grupo de cohomología de $C_\bullet$ con coeficientes en $G$}.

Consideremos
\[\begin{tikzcd}
	\Hom(C_{n-1},G)\arrow[r,"\partial_{n-1}^*"]&\Hom(C_{n-1},G)\arrow[r,"\partial_n^*"]&\Hom(C_n,G)
\end{tikzcd}\]
Y también
\[\begin{tikzcd}
	C_{n-2}\arrow[r,"f"]&G\\
	C_{n-1}\arrow[u,"\partial_{n-1}"]\arrow[ur,"f\partial_{n-1}",swap]
\end{tikzcd}\]
De manera que los elementos en la homología funciones que se anulan en las fronteras, ya que $[g]\in H^{n-1}(C;G)$ para alguna $g:C_{n-1}\to G$ tal que
\begin{align*}
	f\partial_{n-1}:C_{n-1}&\to G\\
	g&\mapsto g\partial_n=0
\end{align*}
\subsection{$\Cohom$ y $\Hom$}
	Los funtores homología y dualizar no conmutan, es decir $H^n(C_\bullet;G)$ y $\Hom(H_n(C_\bullet),G)$ no son iguales.
\begin{ejem}
	Analizar el caso de 
	\[\begin{tikzcd}
		C_\bullet&0\arrow[r]&\Z\arrow[r,"2"]&\Z\arrow[r]&0
	\end{tikzcd}\]
	comparando $H^n(C_\bullet;\Z)$ con $\Hom(H_n(C_\bullet);\Z)$.
	Para calcular la cohomología lo primero que hago es dualizar:
	\[\begin{tikzcd}
		C^*_\bullet&0&\Z\arrow[l,swap,"2"]&\Z\arrow[l]&0\arrow[l]
	\end{tikzcd}\]
	De manera que $H^0(C_\bullet;\Z)=0$ y $H^1(C_\bullet;\Z)=\Z/2\Z$, pero $\Hom(H_n(C_\bullet),\Z)=0$ para toda $n$.
\end{ejem}
Aún así, podemos construir una función
\begin{align*}
	h:H^n(C_\bullet;G)&\to\Hom(H_n(C_\bullet),G)\\
	[g]&\mapsto Z_n/B_n=H_n(C_\bullet)\to G
\end{align*}
donde $g:C_n\to G$ con $g|_{B_n}=0 $. Así que simplemente enviamos a $[g]$ a la restricción $g|_{Z_n}:Z_n/B_n\to G$.
\begin{prop}
	$h$ es suprayectiva. Más aún, exste una función 
	\[\varphi:\Hom(H_n(C_\bullet),G)\to H^n(C_\bullet;G)\]
	tal que $h\varphi=id$.
\end{prop}
\begin{proof}
	Sea $\bar{g}:H_n(C_\bullet)\to G$. Como $H_n(C_\bullet)=Z_n/B_n$, lo que queremos hacer es extender $\bar{g}$ a una función en todo $C_\bullet$. Observemos que tenemos esta sucesión exacta corta:
	\[\begin{tikzcd}
		0\arrow[r]&Z_n\arrow[r]&C_n\arrow[r]&B_{n-1}\arrow[r]&0
	\end{tikzcd}\]
	Como $B_{n-1}\subset C_{n-1}$ y $C_n$ es libre porque \textbf{estamos suponiendo que $C_\bullet$ es un complejo de cadenas de grupos abelianos libres}. Luego, esta sucesión exacta corta se escinde así que $C_n=Z_n\oplus B_n$, y la proyección a $Z_n$ es un mapeo $C_n\to Z_n$. En fin,
	\[\begin{tikzcd}
		Z_n\arrow[r]&H_n(C)\arrow[r,"\bar{g}"]&G\\
		C_n\arrow[u]\arrow[rru,swap,dashed,"g"]
	\end{tikzcd}\]
	Y de hecho$g$ sí representa un elemento en la cohomología, ya que $g|_{B_n}=0$ porque la flecha de $Z_n\to H_n(C)=Z_n/B_n$ es el paso al cociente, así que se pierden los elementos de $B_n$. Además, $\varphi$ es un homomorfismo. Ya además $\varphi h=id$ por la forma en la que fue contruida: no hemos hecho más que extender y luego restringir una función.
\end{proof}
\begin{coro}
	$H^n(C_\bullet,G)\approx\Hom(H_n(C_\bullet),G)\oplus\ker h$.
\end{coro}
\begin{prop}
	$\ker h\approx \Ext^1_\Z(H_{n-1},G)$.
\end{prop}
\begin{proof}
	Esta prueba tiene un truco. Comenzamos por definir el complejo de cadenas de los ciclos,
	\[\begin{tikzcd}
		0\arrow[d]&&&\\
		Z_\bullet\arrow[d,hook]&\cdots\arrow[r]&Z_n\arrow[d,hook]\arrow[r,"0"]&Z_{n-1}\arrow[d,hook]\arrow[r,"0"]&\cdots\\
		C_\bullet\arrow[d]&\cdots\arrow[r]&C_n\arrow[d,hook]\arrow[r,"\partial_n"]&C_{n-1}\arrow[d,hook]\arrow[r,"\partial{n-1}"]&\cdots\\
		B_{\bullet-1}\arrow[d]&\cdots\arrow[r]&B_n\arrow[r,"0"]&B_{n-1}\arrow[r,"0"]&\cdots\\
		0&&&
	\end{tikzcd}\]
	Los homomorfismos frontera son $0$ cuando vemos este complejo de cadenas como subcomplejo de cadenas de $\C_\bullet$. Y algo parecido para $B_{\bullet-1}$. Ahora, la sucesión exacta corta de complejos de cadenas se escinde y al dualizar obtenemos
	\[\begin{tikzcd}
		0\arrow[r]&B_{\bullet-1}^*\arrow[r]&C^*_\bullet\arrow[r]&Z_\bullet^*\arrow[r]&0
	\end{tikzcd}\]
	Ahora sí, aplicamos el teorema fundamental del álgebra homológica para obtener
	\[\begin{tikzcd}
		\cdots\arrow[r]&B_{n-1}^*\arrow[r]&H^n(C_\bullet,G)\arrow[r]&Z_n^*\arrow[r]&B_n^*\arrow[r]&H^{n+1}(C_\bullet,G)\arrow[r]&\cdots
	\end{tikzcd}\]
	\begin{af}
		$Z_n^*\xrightarrow{i_n^*} B_n^*$ dado por $g\mapsto g|_{B_n}$ es el dual de $B_n\hookrightarrow Z_n$.
	\end{af}
	\begin{obs}
		El dual de una inclusión es una restricción.
	\end{obs}
	En la prueba simplemente mostramos que el mapeo $g\mapsto \bar{g}\partial_{n+1}:B_n\to G$ es una restricción.
	\[\begin{tikzcd}
		0\arrow[d]&&0&0&\\
		Z_\bullet\arrow[d,hook]&\cdots\arrow[r]&Z^*_n\arrow[r,"0",red]\arrow[u]&Z^*_{n-1}\arrow[u,red]\arrow[r,"0"]&\cdots\\
		C_\bullet\arrow[d]&\cdots\arrow[r]&C^*_n\arrow[u,red]\arrow[r,"\partial^*_n"]&C_{n-1}\arrow[u]\arrow[r,"\partial^*_{n-1}"]&\cdots\\
		B_{\bullet-1}\arrow[d]&\cdots\arrow[r]&B^*_n\arrow[r,"0"]\arrow[u]&B^*_{n-1}\arrow[u]\arrow[r,"0"]&\cdots\\
		0&&&
	\end{tikzcd}\]
	Perseguimos ese diagrama para construir el diagrama conmutativo
	\[\begin{tikzcd}
		&Z_n\arrow[r,"g"]\arrow[d,hook]&G\\
		B_n\arrow[ur,hook]\arrow[r,hook]&C_n\arrow[ur,"\bar{g}"]\\
		&C_{n+1}\arrow[ul]\arrow[u]
	\end{tikzcd}\]
	\textit{Parece que la $\bar{g}$ cambió de ser una restricción a una extensión...}
	
	Luego
	\[\begin{tikzcd}
		0\arrow[r]&\coker i_{n-1}^*\arrow[r]&H^n(C_\bullet;G)\arrow[r,"h"]&\ker i_{n}^*\arrow[r]&0
	\end{tikzcd}\]
	\begin{af}
		$\ker i^*_n=\Hom(H_n(C_\bullet),G)$
	\end{af}
	\begin{af}
		$\ker h=\coker i_{n-1}^*$.
	\end{af}
	Entonces,
	\[\begin{tikzcd}
		Z_{n-1}^*\arrow[r,"i_{n-1}^*"]&B_{n-1}^*\arrow[r]&\coker i_{n-1}^*\arrow[r]&0
	\end{tikzcd}\]
	Y ahora nada más tomamos
	\[\begin{tikzcd}
		&&&0\\
		0\arrow[r]&B_{n-1}\arrow[r,hook]&Z_{n-1}\arrow[r]\arrow[ur]&H_{n-1}(C_\bullet)\arrow[r]&0
	\end{tikzcd}\]
	Y de aquí que
	\[\begin{tikzcd}
		0\arrow[dr]&&&0\\
		&Z_{n-1}^*\arrow[r,"i_{n-1}^*"]&B_{n-1}^*\arrow[r]\arrow[ur]&\coker i_{n-1}^*\arrow[r]&0
	\end{tikzcd}\]
	Y esa de cuatro no es exacta pero sí concluimos que $\coker i_{n-1}^*=\Ext^1(H_n(C_\bullet),G)$.
\end{proof}
\begin{teo}
	Tenemos
	\[\begin{tikzcd}
		0\arrow[r]&\Ext^1(H_{n-1}(C_\bullet),G)\arrow[r]&H^n(C_\bullet,G)\arrow[r,"h"]&\Hom(H_n(C),G)\arrow[r]&0
	\end{tikzcd}\]
\end{teo}
\end{document}